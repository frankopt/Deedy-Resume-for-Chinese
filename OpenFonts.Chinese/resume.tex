%%%%%%%%%%%%%%%%%%%%%%%%%%%%%%%%%%%%%%%
% Deedy - One Page Two Column Resume
% LaTeX Template
% Version 1.2 (16/9/2014)
%
% Original author:
% Debarghya Das (http://debarghyadas.com)
%
% Original repository:
% https://github.com/deedydas/Deedy-Resume
%
% IMPORTANT: THIS TEMPLATE NEEDS TO BE COMPILED WITH XeLaTeX
%
% This template uses several fonts not included with Windows/Linux by
% default. If you get compilation errors saying a font is missing, find the line
% on which the font is used and either change it to a font included with your
% operating system or comment the line out to use the default font.
% 
%%%%%%%%%%%%%%%%%%%%%%%%%%%%%%%%%%%%%%
% 
% TODO:
% 1. Integrate biber/bibtex for article citation under publications.
% 2. Figure out a smoother way for the document to flow onto the next page.
% 3. Add styling information for a "Projects/Hacks" section.
% 4. Add location/address information
% 5. Merge OpenFont and MacFonts as a single sty with options.
% 
%%%%%%%%%%%%%%%%%%%%%%%%%%%%%%%%%%%%%%
%
% CHANGELOG:
% v1.1:
% 1. Fixed several compilation bugs with \renewcommand
% 2. Got Open-source fonts (Windows/Linux support)
% 3. Added Last Updated
% 4. Move Title styling into .sty
% 5. Commented .sty file.
%
%%%%%%%%%%%%%%%%%%%%%%%%%%%%%%%%%%%%%%%
%
% Known Issues:
% 1. Overflows onto second page if any column's contents are more than the
% vertical limit
% 2. Hacky space on the first bullet point on the second column.
%
%%%%%%%%%%%%%%%%%%%%%%%%%%%%%%%%%%%%%%


\documentclass[]{deedy-resume-openfont}
\usepackage{fancyhdr}
    
\pagestyle{fancy}
\fancyhf{}
    
\begin{document}

%%%%%%%%%%%%%%%%%%%%%%%%%%%%%%%%%%%%%%
%
%     LAST UPDATED DATE
%
%%%%%%%%%%%%%%%%%%%%%%%%%%%%%%%%%%%%%%
\lastupdated

%%%%%%%%%%%%%%%%%%%%%%%%%%%%%%%%%%%%%%
%
%     TITLE NAME
%
%%%%%%%%%%%%%%%%%%%%%%%%%%%%%%%%%%%%%%
\namesection{史}{文哲}{ \urlstyle{same}\href{nwpufrank@gmail.com}{whoami@whoareyou.com} | 176 1043 6835
}

%%%%%%%%%%%%%%%%%%%%%%%%%%%%%%%%%%%%%%
%
%     COLUMN ONE
%
%%%%%%%%%%%%%%%%%%%%%%%%%%%%%%%%%%%%%%

\begin{minipage}[t]{0.25\textwidth} 

%%%%%%%%%%%%%%%%%%%%%%%%%%%%%%%%%%%%%%
%     EDUCATION
%%%%%%%%%%%%%%%%%%%%%%%%%%%%%%%%%%%%%%

\section{教育经历} 
\sectionsep

\subsection{西北工业大学}
\descript{硕士学位,电子与通信工程}
\location{2014.09-2017.04}
\sectionsep

\subsection{青岛理工大学}
\descript{学士学位,主修通信工程}
\location{2010.09-2014.06}
\sectionsep

%%%%%%%%%%%%%%%%%%%%%%%%%%%%%%%%%%%%%%
%     LINKS
%%%%%%%%%%%%%%%%%%%%%%%%%%%%%%%%%%%%%%

\section{链接}
\sectionsep 
Github:// \href{https://github.com/frankopt}{\bf frank.shi} \\
LinkedIn://  \href{https://www.linkedin.com/in/wenzheshi/}{\bf wenzhe.shi} \\

%%%%%%%%%%%%%%%%%%%%%%%%%%%%%%%%%%%%%%
%     COURSEWORK
%%%%%%%%%%%%%%%%%%%%%%%%%%%%%%%%%%%%%%

% \section{修读课程}
% \subsection{Graduate}
% Advanced Machine Learning \\
% Open Source Software Engineering \\
% Advanced Interactive Graphics \\
% Compilers + Practicum \\
% Cloud Computing \\
% Evolutionary Computation \\
% Defending Computer Networks \\
% Machine Learning \\
% \sectionsep

%%%%%%%%%%%%%%%%%%%%%%%%%%%%%%%%%%%%%%
%     SKILLS
%%%%%%%%%%%%%%%%%%%%%%%%%%%%%%%%%%%%%%

\section{技能}
\sectionsep
\subsection{数据库}
\location{主要负责}
Grid Infrastructure \textbullet{} RAC \textbullet{} ACFS \\
\location{工作相关}
ASM \textbullet{} RMAN \textbullet{} PL/SQL \textbullet{} \LaTeX\ \\


\subsection{测试工具}
\location{常用}
Swingbench \textbullet{} AROLTP \textbullet{} Bonnie++  \\
\location{了解}
iPerf \textbullet{} Linpack \textbullet{} FIO  \\
\sectionsep

\subsection{云计算}
\location{了解}
AWS Fundamentals(有AWS课程证书)

%%%%%%%%%%%%%%%%%%%%%%%%%%%%%%%%%%%%%%
%
%     COLUMN TWO
%
%%%%%%%%%%%%%%%%%%%%%%%%%%%%%%%%%%%%%%

\end{minipage} 
\hfill
\begin{minipage}[t]{0.73\textwidth} 

%%%%%%%%%%%%%%%%%%%%%%%%%%%%%%%%%%%%%%
%     EXPERIENCE
%%%%%%%%%%%%%%%%%%%%%%%%%%%%%%%%%%%%%%

\section{工作经历}
\sectionsep
\runsubsection{Oracle(甲骨文)}
\descript{Senior Member of Technical Staff - Software Developer 3.PRODEV.}
\location{2017.07  至今 | 北京}
\vspace{\topsep}
\begin{tightemize}
    \item 主要负责高可⽤性集群数据库(Oracle RAC)的CLUSTERWARE和DATABASE软件的测试工作
    \item 主要对RAC集群产品进行功能/性能/破坏性/压力/兼容性等测试
    \item 解决在测试过程中发现的问题,提供解决⽅案并且帮助美国开发同事进行问题分析和定位
    \item 对开发提供的 patch 进⾏诊断分析及验证并进⾏回归测试
    \item 编写测试文档与测试用例,并实现脚本自动化
\end{tightemize}
\sectionsep

\runsubsection{百度}
\descript{网页搜索运维实习生}
\location{2016.07-2016.09 | 北京}
\begin{tightemize}
\item ⼿机百度、WISE⻚面内容监控运营
\item 监控实时热点动态,后台抓取与动态⼲预内容相结合
\item 后台频次特征查询监控,归纳用户query数据特征
\end{tightemize}
\sectionsep

%%%%%%%%%%%%%%%%%%%%%%%%%%%%%%%%%%%%%%
%     RESEARCH
%%%%%%%%%%%%%%%%%%%%%%%%%%%%%%%%%%%%%%

\section{项目与论文}
\sectionsep
\runsubsection{\href{ACFS Stress and Destructive Test}}
\descript{Owner}
\location{负责ACFS Project的20.1/19.2/19.1/18.3/18.1 release testing}
\begin{tightemize}
    \item 跟踪产品更新,针对新的特性开发针对性的测试,并编写测试计划书,review测试⽤例,主要包括basic functions/long running/destructive/boundary/parallel/stress and performance tests
    \item 负责跟踪项⽬进度,提交测试报告。
    \item 针对客户提交的bug搭建环境进⾏重现,配合验证开发特殊需求的patch,完成相关测试报告
    \item ⽤Perl/Python 实现脚本⾃动化,改进测试⼯具及用例的脚本
    \end{tightemize}
\sectionsep

\runsubsection{\href{ASM/IOS project testing}}
\descript{Teamworker}
\location{负责ASM/IOS project的18.1/18.3 release testing}
\begin{tightemize}
    \item 参与设计新特性的测试⽤例, review测试⽤例并编写测试计划书
    \item 搭建压⼒测试环境,创建DB,安装swingbench/aroltp等工具
    \item 积极跟踪bug fix情况,为开发提供测试环境验证patch
    \end{tightemize}
\sectionsep

\runsubsection{\href{RAC Installation/CVU}}
\descript{Owner}
\location{负责Solaris平台RAC安装、CVU相关测试}
\begin{tightemize}
    \item 在Solaris平台搭建RAC,测试多种安装配置,负责发送每周版本安装结果
    \item 配合support工作,为客户需求构建测试场景,并完成基本功能验证
    \end{tightemize}
\sectionsep

%%%%%%%%%%%%%%%%%%%%%%%%%%%%%%%%%%%%%%
%     OPEN SOURCE

% \section{开源贡献}
% \begin{tabular}{ll}
% \href{https://github.com/moby/moby/commits?author=gaocegege}{\bf moby/moby} & 实现 docker % service ps -q 参数,与 swarmkit 更好集成 \\
% \href{https://github.com/opencontainers/runc/commits?author=gaocegege}{\bf % opencontainers/runc} & 为了修复 \href{https://github.com/moby/moby/issues/27484}{moby/% moby\#27484} 对上游进行的修改 \\
% \href{https://github.com/pingcap/tidb/commits?author=gaocegege}{\bf pingcap/tidb} & 在 % travis 里引入了覆盖率测试; 实现 truncate 函数 \\
% \href{https://github.com/coala/coala-vs-code/commits/master?author=gaocegege}{\bf coala/% coala-vs-code} & Visual Studio Code 上的插件,项目 maintainer \\
% \href{https://github.com/weijianwen/SJTUThesis/commits?author=gaocegege}{\bf weijianwen/% SJTUThesis} & 为学士论文模板添加英文大摘要; 替换版权字体 \\
% \end{tabular}
% \sectionsep
%%%%%%%%%%%%%%%%%%%%%%%%%%%%%%%%%%%%%%

%%%%%%%%%%%%%%%%%%%%%%%%%%%%%%%%%%%%%%
%     AWARDS
%%%%%%%%%%%%%%%%%%%%%%%%%%%%%%%%%%%%%%

\section{爱好及经历} 
\begin{tabular}{rll}
2019       & Promote to Software Developer 3.PRODEV \\
2018	     & 二等奖  & 第十三届全国研究生数学建模竞赛 \\
2017	     & Oracle深圳游泳协会会长,Oracle深圳年会主持人 \\
2016	     & 西北工业大学电子信息学院篮球队主力后卫 \\
2015       & 西北工业大学二等奖学金,西北工业大学校园“微拓”演讲主讲人 \\

\end{tabular}
\sectionsep

%%%%%%%%%%%%%%%%%%%%%%%%%%%%%%%%%%%%%%
%     PUBLICATIONS
%%%%%%%%%%%%%%%%%%%%%%%%%%%%%%%%%%%%%%

% \section{Publications} 
% \renewcommand\refname{\vskip -1.5cm} % Couldn't get this working from the .cls file
% \bibliographystyle{abbrv}
% \bibliography{publications}
% \nocite{*}

\end{minipage} 
\end{document}  \documentclass[]{article}
